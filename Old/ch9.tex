% last modified 11/23/10 by PBS

% last modified 11/2/10 by PB Stark, statistics.berkeley.edu/~stark
\documentclass[12pt]{article}
\topmargin -.5in
\oddsidemargin -.4in
\evensidemargin -.4in
\textwidth 7.0in
\textheight 9in

\title{Statistics 240 Lecture Notes}
\author{P.B. Stark
{\tt statistics.berkeley.edu/$\sim$stark/index.html}}
\newfont{\msbm}{msbm10 at 12pt}
\renewcommand{\baselinestretch}{1.5}
\newcommand{\EE}{\mbox{\msbm E}}
\newcommand{\card}{\mbox{\#}}
\newcommand{\diam}{{\rm diam}}
\newcommand{\sgn}{{\rm sgn}}
\newcommand{\med}{{\rm med}}
\newcommand{\MAD}{{\rm MAD}}
\newcommand{\Prob}{{\bf P}}
\newcommand{\calGbar}{{\bar{{\cal G}}}}
\newcommand{\calA}{{\cal A}}
\newcommand{\calB}{{\cal B}}
\newcommand{\calC}{{\cal C}}
\newcommand{\cC}{{\cal C}}
\newcommand{\calD}{{\cal D}}
\newcommand{\cD}{{\cal D}}
\newcommand{\calF}{{\cal F}}
\newcommand{\calG}{{\cal G}}
\newcommand{\calH}{{\cal H}}
\newcommand{\cH}{{\cal H}}
\newcommand{\calI}{{\cal I}}
\newcommand{\calJ}{{\cal J}}
\newcommand{\calK}{{\cal K}}
\newcommand{\calL}{{\cal L}}
\newcommand{\calM}{{\cal M}}
\newcommand{\calN}{{\cal N}}
\newcommand{\calP}{{\cal P}}
\newcommand{\calQ}{{\cal Q}}
\newcommand{\calR}{{\cal R}}
\newcommand{\calS}{{\cal S}}
\newcommand{\calT}{{\cal T}}
\newcommand{\calU}{{\cal U}}
\newcommand{\calV}{{\cal V}}
\newcommand{\calW}{{\cal W}}
\newcommand{\calX}{{\cal X}}
\newcommand{\cX}{{\cal X}}
\newcommand{\calY}{{\cal Y}}
\newcommand{\calZ}{{\cal Z}}
\newcommand{\bfC}{{\bf C}}
\newcommand{\bfQ}{{\bf Q}}
\newcommand{\bfone}{{\bf 1}}
\def\Real{\hbox{I\kern-.17em\hbox{R}}}
\def\Prob{\hbox{I\kern-.17em\hbox{P}}}
\def\Expect{\hbox{I\kern-.17em\hbox{E}}}
\newcommand{\bfR}{\Real}
\newcommand{\bfRn}{\bfR^n}
\newcommand{\bfRN}{\bfR^N}
\newcommand{\bfRm}{\bfR^m}
\def\bfN{\hbox{I\kern-.17em\hbox{N}}}
\newcommand{\bfZ}{{\bf Z}}
\newcommand{\bfX}{{\bf X}}
\newcommand{\bfx}{{\bf x}}
\newcommand{\bfs}{{\bf s}}
\newcommand{\bfe}{{\bf e}}
\newcommand{\bfhats}{{\bf \hat{s}}}
\newcommand{\bfex}{{\langle \bfe , x \rangle }}
\newcommand{\gbar}{{\bar{g}}}
\newcommand{\Fhatn}{{\hat{F}_n}}
\newcommand{\beq}{\begin{equation}}
\newcommand{\eeq}{\end{equation}}
\newcommand{\Tau}{{\bf T}}
\newcommand{\Eta}{{\cal E}}
\newcommand{\sech}{{\rm sech}}
\newcommand{\ip}[2]{{\langle #1 , #2 \rangle }}
\newcommand{\Choose}[2]{{{\left ( \begin{array}{c} #1\\#2 \end{array} \right )}}}
\newcommand{\linSpan}{{\rm span}}
\newcommand{\dom}{{\rm dom}}
\newcommand{\gto}{{\rm G\^{a}teaux }}
\newcommand{\IF}{{\rm IF}}
\newcommand{\supp}{{\rm supp}}
\newcommand{\Pranki}{{P_{(i)}}}
\newcommand{\Prankj}{{P_{(j)}}}
\newcommand{\Prank}[1]{{P_{(#1)}}}
\newcommand{\Hrank}[1]{{H_{(#1)}}}
\newcommand{\Hranki}{{H_{(i)}}}
\newcommand{\Var}{{\bf Var}}
\newcommand{\Bias}{{\bf Bias}}
\newcommand{\jackProb}[1]{{\Prob_{(#1)}}}
\newcommand{\bootProb}{{\Prob^*}}
\newcommand{\rightWarrow}{{\begin{array}{c} {} \\ \rightarrow \\ W \end{array}}}

\begin{document}
\newtheorem{Theorem}{Theorem}
\newtheorem{Definition}{Definition}
\newtheorem{Exercise}{Exercise}
\newtheorem{Corollary}{Corollary}
\newtheorem{Lemma}{Lemma}
\newtheorem{Proof}{Proof}
\maketitle



\section{Part 9: Robustness and related topics. ROUGH DRAFT}
References:

\noindent
Hampel, F.R., Rousseeuw, P.J., Ronchetti, E.M., and Strahel, W.A., 1986.
{\em Robust Statistics: The approach based on influence functions}, Wiley, NY.
\\[.1in]
Huber, P.J., 1981. {\em Robust Statistics}, Wiley, N.Y.

\subsection{Heuristics}
So far, we have been concerned with nonparametric tests and procedures:
tests and procedures with minimal assumptions about the probability distribution
that gives rise to the data.
Examples of the kinds of assumptions we have made in various places are that the
distribution is continuous, that
the observations are iid, that subjects are randomized into treatment and control,
{\em etc\/}.

We now consider assumptions that are more restrictive, but still less
restrictive than the parametric assumptions common to much of statistical theory.
In a loose sense, an estimator or test is {\em robust\/} if its performance is
good over a neighborhood of distributions including the family in which the
truth is modeled to lie.

Virtually every real data set has some ``gross errors,'' which are
in some sense corrupted measurements.
Data can be transcribed incorrectly,
bits can be corrupted in transmission or storage, cosmic rays can hit satellite-borne
instruments, power supplies can have surges, operators can miss their
morning cups of coffee, {\em etc.}
Hampel {\em et al.\/} (1986) quote studies that find gross errors comprise up to 20\%
of many data sets in physical, medical, agricultural and social sciences, with typical
rates of 6\% or more.

Even  without gross errors, there is
always a limit on the precision with which data are recorded, leading to
truncation or rounding errors that make continuous models for data error
distributions only approximate.
Furthermore, parametric error models rarely are dictated by direct physical arguments;
rather,  the central limit theorem is invoked, or some historical assumption
becomes standard in the field.

One early attempt to deal with gross outliers is due to Sir Harold
Jeffreys, who (in the 1930s) modeled data errors as a mixture of two Gaussian distributions,
one with variance tending to infinity.  
The idea is that the high-variance Gaussian represents a fraction of gross outliers possibly
present in the data; one wants to minimize the influence of such observations
on the resulting estimate or test.
Considerations in favor of fitting by minimizing mean
absolute deviation instead of least squares go back much further.

We will be looking at ways of quantifying robustness and of constructing
procedures whose performance when the model is true is not much worse than
the optimal procedure, but whose performance when the model is wrong (by a little
bit) is not too bad, and is often much better than the performance of the
optimal procedure when the model for which it is optimal 
is wrong by a little bit.

The kinds of questions typically asked in the robustness literature are:
\begin{enumerate}
    \item Is the procedure sensitive to small departures from the model?
    \item To first order, what is the sensitivity?
    \item How wrong can the model be before the procedure produces garbage?
\end{enumerate}
The first issue is that of qualitative robustness; the second is quantitative robustness;
the third is the ``breakdown point.''

\subsection{Resistance and Breakdown Point}
{\em Resistance} has to do with changes to the observed data, rather than to the
theoretical distribution underlying the data.
A statistic is {\em resistant} if arbitrary changes to a few data (such as
might be caused by {\em gross outliers}, or small changes
to all the data (such as might be caused by rounding or truncation),
result in only small changes to the value of the statistic.

Suppose we are allowed to change the values of the observations in the sample.
What is the smallest fraction would we need to change to make the 
estimator take an arbitrary value?
The answer is the {\em breakdown point\/} of the estimator.

For example, consider the sample mean $\bar{X} = \frac{1}{n} \sum_{j=1}^n X_j$
as an estimator of the mean of $F$. 
(The mean can be written as a functional of
the distribution: $T(F) = \int x dF$; $\theta = T(F_\theta)$.)
Corrupting a single observation can make the sample mean take any real value:
the breakdown point is $\frac{1}{n}$.

In contrast, consider the ``$\alpha$-trimmed mean,'' defined as follows:
Let $k_\ell = \lfloor \alpha n \rfloor$ and $ k_h = \lceil (1-\alpha) n \rceil$.
Let $X_{(j)}$ be the $j$th order statistic of the data.
Define
\beq
    \bar{X}_\alpha = \frac{1}{k_h - k_\ell + 1} \sum_{j = k_\ell}^{k_h} X_{(j)} .
\eeq
This measure of location is less sensitive to outliers than the sample mean is:
Its breakdown point is $n^{-1} \min(k_\ell, n-k_h + 1)$.
An alternative, not necessarily equivalent, definition of the $\alpha$-trimmed
mean is through the functional
\beq
    T(F) = \frac{1}{1-2\alpha}\int_\alpha^{1-\alpha} F^{-1}(t) dt.
\eeq
This version has breakdown point  $\alpha$.

We shall make the notion of breakdown point more precise presently; a few definitions
are required.

\begin{Definition}
        The L\`{e}vy distance between two distribution functions $F$ and $G$
        on $\bfR$ is
        \beq
            \lambda(F, G) = \inf\{ \epsilon : F(x-\epsilon) - \epsilon \le G(x) \le F(x+\epsilon) +
            \epsilon, \;\;\forall x \in \bfR \}.
        \eeq
\end{Definition}

The L\'{e}vy distance makes sense for probability distributions on the real line, where
we can define the cdf.
We want to extend this kind of notion to probability distributions on more complicated
sample spaces, including $\bfR^k$.
Ultimately, we want a metric on probability distributions on Polish spaces,
which are defined below.
To do that, we need a bit more abstract structure.

The following definitions are given to refresh your memory.
For more detail, see Rudin, W. (1974). {\em Real and Complex Analysis\/},
2nd edition, McGraw-Hill, N.Y., pages 636--650 of Le Cam, L. (1986). {\em Asymptotic Methods in
Statistical Decision Theory\/}, Springer-Verlag, NY, or Halmos, P.R. (1974). {\em Measure Theory\/},
Springer-Verlag, N.Y.

\begin{Definition}
    A {\em topology\/} $\calV$ on a space $\calX$ is a collection of subsets $V$ of $\calX$ such
    that
    \begin{enumerate}
        \item $\emptyset \in \calV$.
        \item If $V_j \in \calV$, $j = 1, \ldots, n$, then $\cap_{j=1}^n V_j \in \calV$.
        \item For any collection $\{V_\alpha\}_{\alpha \in A}$ of elements of $\calV$,
            $\cup_{\alpha \in A} V_\alpha \in \calV$.
    \end{enumerate}
    The elements of $\calV$ are called {\em open sets\/}, and we call $\calX$ a
    {\em topological space\/}.
    Complements of elements of $\calV$ are called {\em closed sets\/}.
    Every open set $V$ that contains $x \in \calX$ is called a {\em neighborhood\/} of
    $x$.
    The collection of sets $\calB \subset \calV$ is a {\em base\/}
    if for every $x \in \calX$ and every neighborhood
    $V$ of $x$, $x \in B \subset V$ for some $B \in \calB$.
    The space $\calX$ is {\em separable\/} if it has a countable base.
    Let $A$ be a subset of $\calX$.
    The largest open set contained in $A$ is called the
    {\em interior\/} of $A$, denoted $A^\circ$.
    The smallest closed set that contains $A$ is called the {\em closure\/} of
    $A$, denoted $\bar{A}$.
\end{Definition}

\begin{Definition}
    Let $\calX$ be a topological space with topology $\calV$ and let $\calY$ be
    a topological space with topology $\calW$. A mapping $f: \calX \rightarrow \calY$
    is {\em continuous\/} if $f^{-1}(W) \in \calV$ for every $W \in \calW$.
\end{Definition}

That is, a function from one topological space into another is continuous if the
preimage of every open set is an open set.
(The preimage of a set $A \subset \calY$ under the mapping $f: \calX \rightarrow \calY$
is $f^{-1}(A) \equiv \{ x \in \calX : f(x) \in A \}$.
Note that the preimage of a set that contains a single point can be a set
containing more than one point.)

\begin{Definition}
    A collection $\calM$ of subsets of a set $\calX$ is a {\em $\sigma$-algebra\/}
    on $\calX$ if
    \begin{enumerate}
        \item $\calX \in \calM$
        \item $A \in \calM$ implies $A^c \in \calM$
        \item if $A_j \in \calM$, $j = 1, 2, 3, \ldots$, then $\cup_{j=1}^\infty A_j \in \calM$
    \end{enumerate}
    If $\calM$ is a $\sigma$-algebra on $\calX$, then we call $\calX$ a {\em measurable space\/}
    and we call the elements of $\calM$ the {\em measurable sets\/} of $\calX$.
    If $\calX$ is a measurable space with $\sigma$-algebra $\calM$ and $\calY$ is a topological
    space with topology $\calW$ then
    $f: \calX \rightarrow \calY$ is a {\em measurable function\/} if $f^{-1}(W) \in \calM$ for
    every $W \in \calW$.
\end{Definition}

For any collection $\calF$ of subsets of a set $\calX$, there is a smallest $\sigma$-algebra
containing $\calF$.

\begin{Theorem}
        If each member of a sequence $\{f_j\}_{j=1}^\infty$ is a measurable function from
        a measurable space $\calX$ into $[-\infty, \infty]$, then so is
        $\sup_j f_j$ and so is $\limsup_j f_j$.
\end{Theorem}


\begin{Definition}
        A real-valued function $s: \calX \rightarrow \Re^+$ on a measurable space $\calX$ is
        {\em simple\/} if its range consists of finitely many points in $\Re^+$.
        Such a function can be written in the form
        \beq \label{eq:simpleDef}
                s(x) = \sum_{j=1}^n a_j 1_{x \in A_j}
        \eeq
        for a set $\{a_j\}_{j=1}^n \subset \Re^+$ and a collection of subsets
        $\{A_j\}_{j=1}^n$ of $\calX$.
        If $\{A_j\}_{j=1}^n$ are measurable, then $s$ is a measurable simple function.
\end{Definition}

\begin{Definition}
        A {\em positive measure\/} $\mu$ on a measurable space $\calX$ with $\sigma$-algebra
        $\calM$ is a mapping from $\calM \rightarrow \Re^+$ such that
        \begin{enumerate}
                \item $\mu(A) \ge 0$ $\forall A \in \calM$
                \item if $\{A_j\}_{j=1}^\infty$ are pairwise disjoint, then
                \beq
                        \mu(\cup_{j=1}^\infty A_j) = \sum_{j=1}^\infty \mu(A_j).
                \eeq
        \end{enumerate}
        A positive measure $\mu$ for which $\mu(\calX) = 1$ is called a
        {\em probability measure\/}.
        A measurable space on which a measure is defined is called a {\em measure space\/}.
\end{Definition}

\begin{Definition}
        The integral of a measurable simple function $s$ of the form \ref{eq:simpleDef}
        with respect to the measure $\mu$ is
        \beq
                \int_\calX s d\mu \equiv \sum_{j=1}^n a_j \mu(A_j).
        \eeq
\end{Definition}

Integrals of more general measurable functions are defined as limits of integrals
of simple functions.

\begin{Theorem}
        Every nonnegative real-valued measurable function $f$ is the limit of an
        increasing sequence $\{s_n\}$
        of measurable simple functions.
\end{Theorem}

\begin{Definition}
        The integral of a nonnegative measurable function $f$ with respect to the measure $\mu$
        is
        \beq
               \int_\calX f d\mu \equiv \sup_{s \le f, s \mbox{ simple }} \int_\calX s d\mu.
        \eeq
\end{Definition}

The {\em positive part} $f^+$ of a real-valued function $f$ is $f^+(x) \equiv \max(f(x),0)
\equiv f \vee 0$.
The {\em negative part} $f^-$ of a real-valued function $f$ is $f^-(x) \equiv -\min(f(x),0)
\equiv f \wedge 0$.
If $f$ is measurable, so are $|f|$, $f^+$ and $f^-$.

\begin{Definition}
        The integral of a measurable function $f$ with respect to the measure $\mu$
        is
        \beq
               \int_\calX f d\mu \equiv \int_\calX f^+ d\mu - \int_\calX f^- d\mu
        \eeq
        if $\int_\calX |f| d\mu < \infty$.
\end{Definition}

The following three theorems are the workhorses of the theory:

\begin{Theorem}[Lebesgue's Monotone Convergence Theorem.]
        Let $\{f_j\}_{j=1}^\infty$ be a nonnegative sequence of functions such that
        for every $x \in \calX$,
        $f_1(x) \le f=2(x) \le \cdots \le \infty$ and $f_j(x) \rightarrow f(x)$ as
        $j \rightarrow \infty$.
        Then $f$ is measurable and $\int_\calX f_j d\mu \rightarrow \int_\calX f d\mu$.
\end{Theorem}

\begin{Theorem}[Fatou's Lemma.]
        If $\{f_j\}$ is a sequence of nonnegative measurable functions, then
        \beq
                \int_\calX (\liminf_{j \rightarrow \infty} f_j) d\mu \le
                \liminf_{j\rightarrow \infty} \int_\calX f_j d\mu.
        \eeq
\end{Theorem}

\begin{Theorem}[Lebesgue's Dominated Convergence Theorem.]
        Let $\{f_j\}_{j=1}^\infty$ be a sequence of measurable functions such that
                for every $x \in \calX$,
                $f(x) = \lim_j f_j(x)$ exists.
                If there is a function $g$ such that $\int_\calX |g| d\mu < \infty$ and
                $|f_j(x)| \le g(x)$ for all $x \in \calX$ and all $j \ge 1$, then
                $\int_\calX |f| d\mu < \infty$, $\lim_j \int_\calX | f_j - f | d\mu = 0$,
                and $\lim_j \int_\calX f_j d\mu = \int_\calX f d\mu$.
\end{Theorem}

\begin{Definition}
    If $\calX$ is a topological space with topology $\calV$, the {\em Borel $\sigma$-algebra\/}
    $\calB$ on $\calX$ is the smallest $\sigma$-algebra containing $\calV$.
    The elements of $\calB$ are called the {\em Borel sets\/} of $\calX$.
\end{Definition}

\begin{Definition}
    A {\em metric space\/} is a set $\calX$ and a {\em metric\/} function
    $d: \calX \times \calX \rightarrow \bfR^+$ such that
    \begin{enumerate}
        \item  $d(x, x) = 0$
        \item  $d(x, y) = 0$ implies $x = y$
        \item  $d(x, y) = d(y, x)$, for all $x, y \in \calX$ (symmetry)
        \item  $d(x, y) \le d(x, z) + d(y, z)$ for all
               $x, y, z \in \calX$ (triangle inequality)
    \end{enumerate}
    If $d$ satisfies all these axioms but (2), it is called a {\em pseudometric\/}.
\end{Definition}

An {\em open ball\/} with radius $\epsilon \ge 0$ centered at the point $x$ in the metric space $\calX$
is the set $B_\epsilon(x) = \{ y \in \calX : d(x, y) < \epsilon\}$.
If $\calX$ is a metric space and $\calV$ is the collection of all sets in
$\calX$ that are unions of open balls, then $\calV$ is a topology on $\calX$; it
is called the {\em topology induced by $d$\/}.

If $\calV$ is a topology on $\calX$ such that there exists a metric $d$ on $\calX$
for which $\calV$ is the topology induced by $d$, then $\calV$ is {\em metrizable\/}.

\begin{Definition}
    A sequence $\{ x_j \}_{j=1}^\infty$ of elements of a metric space $\calX$ is
    a {\em Cauchy sequence\/} if for every $\epsilon > 0$ there exists an integer
    $N = N(\epsilon)$ such that $d(x_n, x_m) < \epsilon$ if $\min(n, m) > N$.
\end{Definition}

\begin{Definition}
    A sequence $\{ x_j \}_{j=1}^\infty$ of elements of a metric space $\calX$
    {\em converges to $x \in \calX$\/} if
    \beq
        \lim_{j \rightarrow \infty} d(x_j, x) = 0.
    \eeq
\end{Definition}

\begin{Definition}
    A metric space $\calX$ is {\em complete\/} if every Cauchy sequence in $\calX$ converges
    to an element of $\calX$.
\end{Definition}

\begin{Definition}
    A subset $D$ of a metric space $\calX$ is {\em dense\/} if for every element $x$
    of $\calX$ and every $\epsilon > 0$, there exists $y \in D$ such that
    $d(y, x) < \epsilon$.
\end{Definition}

\begin{Definition}
    A metric space $\calX$ is {\em separable\/} if it contains a countable dense subset.
\end{Definition}

\begin{Definition}
    A {\em vector space over the real numbers\/} is an Abelian group $\calX$ of {\em vectors\/}
    with the group operation $(+ : \calX \times \calX \rightarrow \calX,\; (x, y) \mapsto x+y)$,
    together with the operation of scalar multiplication
    ($: \Re \times \calX \rightarrow \calX,\;\; (\lambda, x)
    \mapsto \lambda x$),
    such that for all $\lambda, \mu \in \Re$ and all $x, y \in \calX$,
    \begin{enumerate}
        \item $\lambda(x + y) = \lambda x + \lambda y$
        \item $(\lambda + \mu)x = \lambda x + \mu x$
        \item $\lambda(\mu x) = (\lambda \mu) x$
        \item $1x = x$
        \item $0x = 0$
    \end{enumerate}
\end{Definition}

\begin{Definition}
    A function $\| \cdot \|: \calX \rightarrow \bfR^+$ on a vector space $\calX$
    is a {\em norm\/} if for all $x, y \in \calX$ and all $\lambda \in \bfR$,
    \begin{enumerate}
        \item $\| x \| \ge 0$ (positive semidefinite)
        \item $\| x \| = 0$ implies $x = 0$ (with (1), this means the norm is positive definite)
        \item $\| \lambda x \| = | \lambda | \| x \|$ (positive scalar homogeneity)
        \item $\| x + y \| \le \|x\| + \|y\|$ (triangle inequality)
    \end{enumerate}
    If $\| \cdot \|$ satisfies all but (2), it is a {\em seminorm\/}.
    If $\| \cdot \|$ is a norm on a vector space $\calX$, we call $\calX$ a
    normed (linear) vector space.
\end{Definition}

A normed linear vector space is a metric space under the metric $d(x, y) \equiv \| x-y\|$.

\begin{Definition}
    A {\em Banach space\/} is a complete normed linear vector space.
\end{Definition}

\begin{Definition}
    A {\em Polish space} $\calX$ is a separable
    topological space whose topology is metrizable
    by a metric $d$ with respect to which $\calX$ is complete.
\end{Definition}

Equivalently, a Polish space is a topological space that is homeomorphic
to a separable Banach space.

Examples of Polish spaces include $\bfR^k$.
Let $\calM$ denote the space of probability measures on the Borel $\sigma$-algebra
$\calB$ of subsets of $\calX$. 
(The Borel $\sigma$-algebra is the smallest $\sigma$-algebra containing
all the open sets in $\calX$.)
Let $\calM'$ denote the set of finite signed measures on  $(\calX, \calB)$; this is
the linear space of measures generated by $\calM$.
The measures in $\calM'$ are {\em regular} in the sense that for every $F \in \calM'$,
\beq
    \sup_{C \subset A;\;C \mbox{ compact }} F(C) = F(A) = \inf_{G \supset A; \; G \mbox{ open }}
    F(G).
\eeq
The weak-star topology in $\calM'$ is the weakest topology for which the functional
\beq
    \int \psi dF
\eeq
is continuous for every continuous, bounded function $\psi: \calX \rightarrow \bfR$.

In this section, we assume that the space $\calX$ of possible data
is a Polish space, and that all measures are defined on $\calB$.
An overbar  ({\em e.g.\/}, $\bar{A}$) denotes topological closure,
the superscript $\circ$ ({\em e.g.\/}, $A^\circ$) denotes the topological interior,
and the superscript $c$ will denote complementation
($A^c = \{ x \in \calX: x \not \in A\}$).


\begin{Definition}
    For any subset $A$ of a metric space $\calX$ with metric $d: \calX \times \calX
    \rightarrow \bfR^+$, the {\em closed $\epsilon$-neighborhood of $A$} is
    \beq
        A^\epsilon = \{ x \in \calX : \inf_{a \in A} d(x,a) \le \epsilon \}.
    \eeq
\end{Definition}
It will be important presently that
\beq
    A^\epsilon = \bar{A}^\epsilon = \overline{A^\epsilon} =
    \overline{\bar{A}^\epsilon}.
\eeq

\begin{Definition}
    The Prohorov distance between two measures $F$ and $G$ defined on a common
    algebra $\calA$ of subsets of a metric space $\calX$ is
    \beq
        \pi(F,G) = \inf \{ \epsilon \ge 0 : F(A) \le G(A^\epsilon) + \epsilon,\;  \;\forall A \in \calA \}
    \eeq
\end{Definition}

Expanding the events by $\epsilon$ to form $A^\epsilon$
corresponds to the measure $G$ being ``shifted'' slightly
from $F$, for example, by rounding.
The addition of $\epsilon$ corresponds to
a fraction $\epsilon$ of the observations being from a completely
different distribution.

We shall verify that the Prohorov distance really is a metric if the sample
space $\calX$ is a Polish space.
Clearly, it is nonnegative, and $\pi(F,F) = 0$.
We need to show symmetry, the triangle inequality, and that $\{\pi(F,G) = 0 \}\Rightarrow
\{F = G\}$.   The following proof follows that in Huber (1981).

\noindent
{\bf Symmetry.}
This will follow immediately if we can show
that if $F(A) \le G(A^\epsilon) + \epsilon$ for all $A \in \calA$, then
$G(A) \le F(A^\epsilon) + \epsilon$ for all $A \in \calA$.
Recall that because $\calA$ is an algebra, if $A \in \calA$, then $A^c \in \calA$
as well.
Take any $\delta > \epsilon$, and consider $A = (B^\delta)^c = B^{\delta c}$ for any
$B \in \calA$.
Note that $A \in \calA$, so by the premise,
\beq
    F(B^{\delta c}) \le G(B^{\delta c\epsilon}) + \epsilon ,
\eeq
or
\begin{eqnarray}
    1 - F(B^\delta) &\le& 1 - G(B^{\delta c \epsilon c}) + \epsilon \\
    G(B^{\delta c \epsilon c}) &\le & F(B^\delta) + \epsilon.
\end{eqnarray}
However, $B \subset B^{\delta c \epsilon c} $, as we shall see.
This statement is equivalent to $B^{\delta c \epsilon} \subset B^c$.
This is essentially immediate from $\delta > \epsilon$: if $x \in B^{\delta c \epsilon}$,
then $\exists y \not \in B^\delta$ s.t. $d(x,y) < \epsilon$ (typo in Huber here).
Thus $x \in B^c$, because otherwise $d(x, y) > \delta > \epsilon$.
Thus
\beq
    G(B) \le G(B^{\delta c \epsilon c}) \le  F(B^\delta) + \epsilon .
\eeq
But $B^\epsilon = \cap_{\delta > \epsilon} B^{\delta}$, so the result follows.
\\[2ex]

\noindent
{\bf Positive definite.}
To show that $(\pi(F,G) = 0) \Rightarrow (F = G)$, note that
the closure of $A$ is $\bar{A} = \cap_{\epsilon > 0} A^\epsilon$.
Thus  $\pi(F,G) = 0 $ implies $F(A) \le G(A)$ and $G(A) \le F(A)$ for all
closed sets $A \in \calA$.
\\[2ex]

\noindent
{\bf Triangle inequality.}
If $\pi(F, G) \le \epsilon$ and $\pi(G, H) \le \delta$, then for every $A \in \calA$,
\beq
    F(A) \le G(A^\epsilon) + \epsilon \le H((A^\epsilon)^\delta) + \epsilon +
    \delta.
\eeq
But by the triangle inequality for the metric $d$ on $\calX$,
$ (A^\epsilon)^\delta  \subset A^{\epsilon+\delta}$, so we are done.

Note that the Prohorov distance between $\hat{F}_n$ and the ``contaminated''
empirical distribution one gets by changing $k$ of the data by an
arbitrary amount is $\frac{k}{n}$.

\begin{Theorem}(Strassen, 1965; see Huber Thm 2.3.7.)
    The following are equivalent:
    \begin{enumerate}
        \item
            $F(A) \le G(A^\delta) + \epsilon$ for all $A \in \calB$
        \item
            There are (possibly) dependent $\calX$-valued random variables $X$, $Y$  such that
            $\calL(X) = F$, $\calL(Y) = G$, and $\Prob\{d(X,Y) \le \delta\} \ge 1 - \epsilon$.
            (Here, $\calL(X)$ denotes the probability law of $X$, {\em etc.})
    \end{enumerate}
\end{Theorem}

\begin{Definition}
Suppose that the distance function $d$ on $\calX \times \calX$ is bounded by one
(one can replace $d$ by $d(x,y)/(1 + d(x, y))$ to make this so). The
{\em bounded Lipschitz metric} on $\calM$ is
\beq
    d_{BL}(F,G) = \sup_{\psi: |\psi(x) - \psi(y)| \le d(x, y)} \left |
    \int \psi  dF - \int \psi dG \right | .
\eeq
\end{Definition}

The bounded Lipschitz metric is truly a metric on $\calM'$.


\begin{Theorem}
    The set of regular Borel measures $\calM'$ on a Polish space $\calX$ is itself
    a Polish space with respect to the weak topology, which is
    metrizable by the Prohorov metric and by the bounded Lipschitz metric.
\end{Theorem}

Consider a collection of probability distributions indexed by $\epsilon$, such as
the Prohorov neighborhood
\beq
    \calP_\pi(\epsilon; F) = \{ G \in \calM : \pi(F, G ) \le \epsilon \}
\eeq
or the ``gross error contamination neighborhood'' (not truly a neighborhood in the
weak topology)
\beq
    \calP_{\mbox{gross error}}(\epsilon; F) = \{ G \in \calM : G = (1-\epsilon)F + \epsilon H,
    H \in \calM \} .
\eeq


Let $M(G, T_n)$ denote the median of the distribution of $T_n(G) - T(F)$.
Let $A(G, T_n)$ denote some fixed percentile of the distribution of
$|T_n(G) - T(F)|$ .
If $T(F_\theta) = \theta, \;$ $\forall \theta \in \Theta$, we say that $T$ is
{\em Fisher consistent\/} for $\theta$.


\begin{Definition}
    Consider a sequence $\{ T_n \}$ of estimators that is Fisher consistent and
    converges in probability to a functional statistic $T$.
    The {\em maximum bias of $\{ T_n\}$ at $F$ over the collection
    $\calP(\epsilon)$} is
    \beq
        b_1(\epsilon) = b_1(\epsilon, \calP, F) =
        \sup_{G \in \calP(\epsilon)} |T(G) - T(F) | .
    \eeq
    The {\em maximum asymptotic bias of $\{ T_n\}$ at $F$ over the collection
    $\calP(\epsilon)$} is
    \beq
        b(\epsilon) = b(\epsilon, \calP, F) =  \lim_{n \rightarrow \infty}
        \sup_{G \in \calP(\epsilon)} |M(G, T_n) | .
    \eeq
\end{Definition}

If $b_1$ is well defined, $b(\epsilon ) \ge b_1(\epsilon)$.
Note that for the gross-error model and for the L\`{e}vy and Prohorov distances,
$b(\epsilon) \le b(1)$, because the set $\calP(1) = \calM$.

\begin{Definition}
    (Following Huber, 1981.)
    For a given collection $\calP(\epsilon)$ of distributions indexed by $\epsilon \ge 0$,
    the {\em asymptotic breakdown point} of $T$ at $F$ is
    \beq
        \epsilon^* \equiv \epsilon^*(F, T, \calP(\cdot)) =
        \sup \{ \epsilon : b(\epsilon, \calP(\epsilon), F) < b(1) \} .
    \eeq
\end{Definition}


\begin{Definition}
    (Following Hampel {\em et al.}, 1986.)
    The {\em breakdown point} of a sequence of estimators $\{ T_n \}$ of a parameter
    $\theta \in \Theta$
    at the distribution $F$
    is
    \beq
        \epsilon^*(T_n, F) \equiv \sup \{\epsilon \le 1 : \exists K_\epsilon \subseteq
        \Theta, \; K_\epsilon \mbox{ compact, s.t. }
        \pi(F,G) < \epsilon \Rightarrow G(\{T_n \in K_\epsilon \}) \rightarrow 1
        \;\mbox{ as } n \rightarrow \infty \} .
    \eeq
\end{Definition}

That is, the breakdown point is the largest Prohorov distance from $F$ a distribution
can be, and still have the estimators almost surely take values in some
compact set as $n \rightarrow \infty$.

\begin{Definition}
    The {\em finite-sample breakdown point of the estimator $T_n$ at $x = (x_j)_{j=1}^n$ }
    is
    \beq
        \epsilon^*(T_n, x) \equiv \frac{1}{n}  \max \left \{ m : \max_{i_1, \ldots, i_m}
        \sup_{y_1, \ldots, y_m \in \calX} | T_n(z_1, \ldots, z_n)| < \infty \right \},
    \eeq
    where
    \beq
        z_j = \left \{ \begin{array}{ll}
        x_j, & j \not \in \{ i_k \}_{k=1}^m \\
        y_k, & j = i_k \mbox{ for some } k.
        \end{array}
        \right .
    \eeq
\end{Definition}

This definition makes precise the notion that corrupting
some fraction of the measurements can corrupt the value of the statistic
arbitrarily.
Note that the finite-sample breakdown point is a function not of a distribution,
but of the sample and the estimator.
Typically, its value does not depend on the sample.
It is this ``breakdown point'' that we saw was zero for the sample mean; it is
a measure of {\em resistance}, not {\em robustness}.

\begin{Definition}
    A sequence of estimators $\{T_n\}$ is {\em qualitatively robust at $F$} if for
    every $\epsilon > 0$, $\exists \delta >0$ such that for all $G \in \calM$ and
    all $n$,
    \beq
        \pi(F,G) < \delta \Rightarrow \pi(\calL_F(T_n), \calL_G(T_n)) < \epsilon .
    \eeq
    That is, $\{T_n\}$ is qualitatively robust at $F$ if the distributions of $T_n$ are
    equicontinuous w.r.t. $n$.
\end{Definition}

\subsection{The Influence Function}
The next few sections follow Hampel {\em et al.} (Ch2) fairly closely.
Consider an estimator of a real parameter $\theta \in \Theta$,
where $\Theta$ is an open convex subset of $\bfR$, based on a sample $X_n$
of size $n$.
The sample space for each observation is $\calX$.
We consider a family $\calF$ of distributions $\{ F_\theta : \theta \in \Theta \}$,
which are assumed to have densities $\{ f_\theta : \theta \in \Theta \}$ with respect
to a common dominating measure.

In our treatment of the bootstrap, we considered problems in which it suffices
to know the empirical distribution of the data: the estimator
could be assumed to be a function of the empirical distribution---at least
asymptotically.
We do the same here.
Consider estimators $\hat{\theta} = T_n(\hat{F}_n)$ that asymptotically can
be replaced by functional estimators.
That is, either $T_n(\hat{F}_n) = T(\hat{F}_n)$ for all $n$, or there is a functional
$T:  \dom(T) \rightarrow \bfR$
such that if the components of $X_n$ are iid $G$,
with $G \in \dom(T)$, then
\beq
    T_n (\hat{G}_n) \rightarrow T(G)
\eeq
in probability as $n \rightarrow \infty$.
$T( G)$ is the asymptotic value of $\{ T_n \}$ at $G$.
We assume that $T(F_\theta) = \theta,$ $\forall \theta \in \Theta$
(this is Fisher consistency of the estimator $T$).


\begin{Definition}
    A functional $T$ defined on probability measures is G\^{a}teaux differentiable at
    the measure $F$ in $\dom(T)$ if there exists a function $a: \calX \rightarrow \bfR$ s.t.
    $\forall G \in \dom(T)$,
    \beq
        \lim_{t \rightarrow 0} \frac{T((1-t)F + tG) - T(F)}{t} = \int a(x) dG(x).
    \eeq
    That is,
    \beq
        \frac{\partial}{\partial t}   T((1-t)F + tG) |_{t=0} =   \int a(x) dG(x).
    \eeq
\end{Definition}

G\^{a}teaux differentiability is weaker than Fr\'{e}chet differentiability.
Essentially,  \gto  differentiability at $F$ ensures that the directional derivatives
of $T$ exist in all directions that (at least infinitesimally) stay in the domain
of $T$.

Let $\delta_x$ be a point mass at $x$.

\begin{Definition}
    The {\em influence function} of $T$ at $F$ is
    \beq    \label{eq:IFdef}
        \IF(x; T, F)  \equiv \lim_{t \rightarrow 0}  \frac{T((1-t)F + t\delta_x) - T(F)}{t}
    \eeq
    at the points $x \in \calX$ where the limit exists.
\end{Definition}

The influence function is similar to the \gto derivative,
but it can exist even when the \gto derivative does not, because the
set of directions it involves is not as rich.
The influence function gives the effect on $T$ of an infinitesimal perturbation to the data
at the point $x$.
This leads to a ``Taylor series'' expansion of $T$ at $F$:
\beq
    T(G) = T(F) + \int \IF(x; T, F) d(G-F)(x) + \mbox{ remainder} .
\eeq
(Note that $\int \IF(x; T, F) dF(x) = 0$.)

\subsubsection{Heuristics using $\IF(x; T, F)$}

Consider what happens for large sample sizes.
The empirical distribution $\hat{F}_n$ tends to the theoretical
distribution $F$, and
$T_n(\hat{F}_n)$ tends to $T(\Fhatn)$.
We will use $\Fhatn$ to denote both the empirical cdf and the
empirical measure, so $\Fhatn = \frac{1}{n}\sum_{j=1}^n \delta_{X_j}$.
Thus
\beq
    (T_n(\Fhatn) - T(F) ) \approx \frac{1}{n} \sum_{j=1}^n \IF(X_j; T, F) + \mbox{ remainder},
\eeq
or
\beq
    \sqrt{n}  (T_n(\Fhatn) - T(F) ) \approx
        \frac{1}{\sqrt{n}} \sum_{j=1}^n \IF(X_j; T, F) + \mbox{ remainder}.
\eeq
Now $\{ \IF(X_j; T, F) \}_{j=1}^n$ are $n$ iid random variables with mean zero
and finite variance, so
their sum is asymptotically normal.  If the remainder vanishes asymptotically
(not that easy to verify, typically, but often true), then
$\sqrt{n}(T_n(\Fhatn) - T(F))$ is also asymptotically normal, with asymptotic variance
\beq    \label{eq:IFasympVar}
    V(T,F) = \int \IF(x; T, F)^2 dF(x).
\eeq

When the relationship holds, an asymptotic form of the Cram\'{e}r-Rao bound relates
asymptotic efficiency to the influence function.
Suppose $T$ is Fisher consistent.
Recall that the Fisher information at $F_{\theta_0}$ is
\beq
    I(F_{\theta_0} ) = \int
    \left ( \left . \frac{\partial}{\partial \theta} \ln f_\theta(x) \right |_{\theta_0}
    \right )^2 dF_{\theta_0} .
\eeq
The Cram\'{e}r-Rao information inequality says that
(under regularity conditions on $\{F_\theta\}_{\theta \in \Theta}$,
including being dominated by a common measure $\mu$, sharing a common support, and
having densities $f_\theta (x)$ w.r.t. $\mu$ that are differentiable in $\theta$)
for any statistic $T$ with $\EE_\theta T^2 < \infty$ for which
\beq
    \frac{d}{d\theta}\EE_\theta T = \int \frac{\partial}{\partial \theta} T f_\theta \mu(dx),
\eeq
\beq
    V(T, F_\theta) \ge \frac{\left [ \frac{\partial}{\partial \theta} \EE_\theta T \right ]^2}{I(F_\theta)}.
\eeq
Thus $T$ is asymptotically efficient only if
\beq
    \IF(x; T, F) = I(F_{\theta_0})^{-1} \frac{\partial}{\partial\theta}(\ln
    f_\theta(x))|_{\theta_0}.
\eeq

\subsubsection{Relation to the Jackknife}
There is an asymptotic connection between the jackknife estimate of variance and
the influence function as well.
Recall that
for a functional statistic $T_n = T(\hat{F}_n)$, we define the $j$th pseudovalue by
\beq
    T_{nj}^* = nT(\hat{F}_n) - (n-1) T(\hat{F}_{(j)}),
\eeq
where $\hat{F}_{(j)}$ is the cdf of the data with the $j$th datum omitted.
The jackknife estimate is
\beq
    T_n^* = \frac{1}{n} \sum_{j=1}^n T_{nj}^*.
\eeq
The (sample) variance of the pseudovalues is
\beq
    V_n = \frac{1}{n-1} \sum_{j=1}^n (T_{nj}^* - T_n^*)^2.
\eeq
The jackknife estimate of the variance of $T_n$ is $\frac{1}{n}V_n$.
Plugging into the definition \ref{eq:IFdef} and taking $t = \frac{-1}{n-1}$ gives
\begin{eqnarray}
    \frac{n-1}{-1} \left ( T\left ( (1 - \frac{-1}{n-1})\hat{F}_n
    + \frac{-1}{n-1} \delta_{x_j}   \right )  - T(\hat{F}_n) \right ) &=&
    (n-1) [ T(\hat{F}_n) - T(\hat{F}_{(j)}) ] \nonumber \\
    &=& T_{nj}^* - T(\hat{F}_n).
\end{eqnarray}
(Note that $(1 - \frac{-1}{n-1})\hat{F}_n + \frac{-1}{n-1} \delta_{x_j} = \hat{F}_{(j)}$.)
The jackknife pseudovalues thus give an approximation to the influence function.

\subsubsection{Robustness measures defined from $\IF(x; T, F)$}
The {\em gross error sensitivity of $T$ at $F$} is
\beq
    \gamma^* = \gamma^*(T, F) = \sup_{\{x: \IF(x; T, F) \mbox{ exists}\}} | \IF(x; T, F) |.
\eeq
This measures the  maximum change to $T$ a small perturbation to $F$ at a point can
induce, which is a bound on the asymptotic  bias of $T$ in a neighborhood of $F$.
If $\gamma^*(T,F) < \infty$, $T$ is {\em B-robust} at $F$ (B is for bias).
For Fisher-consistent estimators, there is typically a minimum possible value of
$\gamma^*(T, F)$, leading to the notion of {\em most B-robust} estimators.
There is usually a tradeoff between efficiency and B-robustness: for a given upper
bound on $\gamma^*$, there is a most efficient estimator.

The gross error sensitivity measures what can happen when the difference between what is
observed and $F$ can be anywhere (perturbing part of an observation by an
arbitrary amount).
There is a different notion of robustness related to
changing the observed values slightly.
The (infinitesimal) effect of moving an observation from $x$ to $y$   is
$\IF(y; T, F) - \IF(x; T, F)$.
This can be standardized by the distance from $y$ to $x$
to give the {\em local shift sensitivity}
\beq
    \lambda^* = \lambda^*(T, F) =
    \sup_{\{x \ne y : \IF(x; T, F) \mbox{ and } \IF(y; T, F) \mbox{ both exist}\}}
    \frac{| \IF(y; T, F) - \IF(x; T, F)|}{| y - x|} .
\eeq
This is the Lipschitz constant of the influence function.
(A real function $f$ with domain $\dom(f)$
is {\em Lipschitz continuous at $x$} if there exists a constant
$C > 0$ such that $|f(x) - f(y)| \le C | x-y|$ for all $y \in \dom(f)$.
A real function $f$ is Lipschitz continuous if it is Lipschitz continuous at every
$x \in \dom(f)$.
The {\em Lipschitz constant\/} of a Lipschitz-continuous function $f$ is
the smallest $C$ such that $|f(x) - f(y)| \le C | x-y|$ for all $x, y \in \dom(f)$.
These definitions extend to functions defined on metric spaces, and to bounding 
$|f(x) - f(y)|$ by a multiple of $| x - y |^\nu$ with $\nu \ne 1$.)

Another measure of robustness involving the influence function is the
{\em rejection point}
\beq
    \rho^* = \rho^*(T, F) = \inf \{ r > 0 : \IF(x; T, F) = 0, \;\;\forall |x| > r \}.
\eeq
This measures how large an observation must be before the estimator ignores it
completely.
If very large observations are almost certainly
gross errors, it is good for $\rho^*$ to be finite.

\subsubsection{Examples}
Suppose $X \sim N(\theta, 1)$; $\theta \in \Theta = \bfR$; $\theta_0 = 0$;
$F = \Phi$;
$T_n(X_n) = \bar{X} = \frac{1}{n} \sum_{j=1}^n X_j$;
\beq
    T(G) \equiv \int x dG(x).
\eeq
The functional $T$ is Fisher-consistent.
Calculating $\IF$ gives
\begin{eqnarray}
    \IF(x; T, F)  &=& \lim_{t \rightarrow 0}
    \frac{\int u d((1-t) \Phi + t \delta_x ) (u) - \int u d\Phi(u)}{t} \nonumber \\
    &=& \lim_{t \rightarrow 0 } \frac{(1-t) \int u d\Phi(u) + t \int u d\delta_x(u) - \int u
    d\Phi(u)}{t} \nonumber \\
    &=& \lim_{t \rightarrow 0} \frac{tx}{t} \nonumber \\
    &=& x.
\end{eqnarray}
The Fisher information of the standard normal with unknown mean is $I(\Phi) =1$, so
\beq
    \int \IF^2(x; T, \Phi) d\Phi = 1 = I^{-1}(\Phi),
\eeq
and $\IF \propto (\partial/\partial \theta) (\ln f_\theta)|_0$.
The arithmetic mean is the MLE and is asymptotically efficient.
The gross-error sensitivity is $\gamma^* = \infty$, the local shift sensitivity
is $\lambda^* = 1$, and the   rejection point is $\rho^* = \infty$.

Suppose $\calX = \{0, 1, 2, \ldots \}$, $\lambda(A) = \#A$,
$\{X_j\}_{j=1}^n \mbox{ iid } \mbox{Poisson}(\theta)$, $\theta \in \Theta = (0, \infty)$.
Then $f_\theta(k) = e^{-\theta} \theta^k/k!$.
The MLE of $\theta$ is
\begin{eqnarray}
   \hat{\theta} &=& \arg \max_{\eta \in (0, \infty)} \prod_{j=1}^n f_\eta(X_j) \nonumber \\
                &=& \arg \max_\eta e^{-n\eta} \prod_j \eta^{X_j}/X_j! \nonumber \\
                &=& \arg \max_\eta e^{-n\eta} \eta^{n\sum_j X_j} \prod_j 1/X_j! \nonumber \\
                & \equiv & \max_\eta L(\eta | \{X_j\}_{j=1}^n).
\end{eqnarray}
Stationarity requires
\begin{eqnarray}
   0 &=& \partial/\partial \eta L(\eta | \{X_j\}_{j=1}^n) |_{\hat{\theta}} \nonumber \\
     &\propto& e^{-n{\hat{\theta}}} \sum_j X_j {\hat{\theta}}^{\sum_j X_j - 1} - n e^{-n{\hat{\theta}}} {\hat{\theta}}^{\sum_j X_j}.
\end{eqnarray}
Which reduces to
\beq
	\sum_j X_j {\hat{\theta}}^{-1} - n = 0; \mbox{ i.e., } \hat{\theta} = \frac{1}{n} \sum_j X_j = \bar{X}.
\eeq
This can be written as the functional estimator $T(\hat{F}_n)$ where $T$ is defined by
\beq
	T(F) \equiv \int_\calX x f(x) \lambda(dx)
\eeq
This estimator is clearly Fisher consistent because
\beq
	T(F_\theta) = \sum_{k=0}^\infty k e^{-\theta} \theta^k/k! = \theta, \;\; \forall \theta \in \Theta.
\eeq
The influence function is defined only at points $x \in \calX$.
The functional $T$ is linear, so
\begin{eqnarray}
	\IF(x; T, F_\theta) &\equiv& \lim_{t \rightarrow 0} 
	    \frac{T((1-t)F_\theta + t \delta_x)) - T(F_\theta)}{t} \nonumber \\
	    & = & \lim_{t \rightarrow 0} 
	    \frac{(1-t) T(F_\theta) + t T(\delta_x) - T(F_\theta)}{t} \nonumber \\
	    & = & x - \theta, \;\; x \in \calX.
\end{eqnarray}

\subsection{$M$-estimators}
We shall consider in more detail what is needed for an $M$-estimator
to be robust.
An $M$-estimator is one of the form
\beq \label{eq:MestDef}
    T_n(X) = \arg \min_{U_n} \sum_{j=1}^n \rho(X_j, U_n),
\eeq
where $\rho$ is a function on $\calX \times \Theta$.
$M$-estimators generalize maximum likelihood estimators.
Maximum likelihood estimators minimize the negative log-likelihood, which
for independent data would be of the form just given, with
$\rho(X_j, U_n) = -\log f(x; \theta)$.
If $\rho$ is differentiable in $\theta$, with
\beq
    \psi(x, \theta) = \frac{\partial}{\partial\theta}\rho(x, \theta),
\eeq
then $T_n$ is a stationary point, so it satisfies
\beq  \label{eq:MestDerivDef}
    \sum_{j=1}^n \psi(X_j, T_n) = 0.
\eeq
An estimator that can be characterized as the solution of an 
equation in the form of either
\ref{eq:MestDef} or \ref{eq:MestDerivDef} is an {\em $M$-estimator}.
The two are not quite equivalent:
If $\rho$ is not differentiable, there is no corresponding $\psi$.
Conversely, $\psi$ need not be the partial derivative of some $\rho$.
An $M$-estimator is characterized by (and identified with) $\rho$ or $\psi$.

Consider the second form, based on $\psi$.
Define $T$ implicitly as the functional for which
\beq \label{eq:implicitTdefM}
    \int \psi(x, T(G)) dG = 0
\eeq
for all $G$ for which the integral exists; then the estimator is $T(\hat{F}_n)$.
We calculate the $\IF$ at $F$ of an $M$-estimator of this form.
Define $F_{t,x} \equiv (1-t)F + t\delta_x$.
We want 
\beq
    T' \equiv \lim_{t \rightarrow 0} \frac{T(F_{t,x}) - T(F)}{t}.
\eeq
Assume we can interchange
integration and differentiation;
then we can differentiate \ref{eq:implicitTdefM}
to get
\beq
    \int \psi(x, T(F)) d(\delta_x - F) + \int \left . \frac{\partial}{\partial\theta}
    \psi(x, \theta) \right |_{T(F)} dF \cdot \left .
    \frac{\partial}{\partial t} T(F_{t,x}) \right |_{t=0} = 0.
\eeq
This yields
\beq \label{eq:IFfromPsi}
    \IF(x; \psi, F) = \frac{\psi(x, T(F))}{- \int
    \left . \frac{\partial}{\partial\theta}  \psi(y,\theta)\right |_{T(F)} dF(y)},
\eeq
if the denominator is not zero.
It turns out that even when $\psi$ is not smooth, equation \ref{eq:IFfromPsi}
holds.

The estimator given by $\psi$ is $B$-robust iff $\psi(\cdot, T(F))$ is
bounded.
The asymptotic variance of the $M$-estimate defined by $\psi$ is
\beq
    V(T, F) = \frac{\int \psi^2(x, T(F)) dF(x)}{\left ( \int \left . \partial \psi(y, \theta)/\partial \theta
    \right |_{T(F)} dF(y) \right )^2}.
\eeq

The maximum likelihood estimate (MLE) is a special case of an $M$-estimate,
taking
\beq
    \rho(x, \theta) = -\ln f_\theta(x).
\eeq
The $\IF$ of the MLE is
\beq
    \IF(x; T, F_\theta) =
    \frac{\left . \partial \ln f_\gamma / \partial \gamma \right |_{\theta}}{V(T, F_\theta)}.
\eeq
The asymptotic variance is in this case the reciprocal of the Fisher information $J(F_\theta)$,
which shows the asymptotic efficiency of the MLE.

The {\em maximum likelihood scores function\/} $s(x, \theta)$ is
\beq
    s(x, \theta) \equiv \left . \frac{\partial \ln f_\gamma (x)}{\partial \gamma} \right |_{\theta};
\eeq
the MLE solves
\beq
    \sum_{j=1}^n s(X_j, \mbox{MLE}_n) = 0.
\eeq

For estimates of location parameters,  $\psi$ typically depends on $x$ and $\theta$
only through their difference:
$\psi(x; \theta) = \psi(x-\theta)$.
For the estimate to be Fisher-consistent, we need $\int \psi dF = 0$;
then
\beq
    \IF(x; F, T) = \frac{\psi(x - T(F))}{\int \psi'(x - T(F)) dF}.
\eeq
The influence function is proportional to $\psi$.
(We might as well take $\psi$ to be the $\IF$.)

Therefore, $M$-estimates have finite gross error sensitivity only if
$\psi$ is bounded, and have a finite rejection point only if $\psi$
{\em redescends\/} to zero for large values of its argument.
For the mean, this does not occur.

The asymptotic variance of such an $M$-estimate of a location parameter is
\beq
    V(\psi, F) = \frac{\int \psi^2 dF}{\left ( \int \psi' dF \right )^2}.
\eeq


\subsubsection{Robustness of $M$-estimates}
See Huber, Ch3 for more details; this is drawn from there.

Let's calculate $b_1(\epsilon)$ for the L\'{e}vy metric for an $M$-estimate
of location, with $\psi(x;t) = \psi(x-t)$, with $\psi$ monotonically increasing.
We will use $\lambda$ to denote something new in this section, so let $d_L(F,G)$
denote the L\'{e}vy distance between
distributions (or cdfs) $F$ and $G$.
Accordingly, we take $\calP(\epsilon) \equiv \{ G : d_L(F,G) \le \epsilon \}$.
Assume that $T(F) = 0$.
Define
\beq
    b_+(\epsilon) = \sup_{G \in \calP(\epsilon)}T(G)
\eeq
and
\beq
    b_-(\epsilon) = \inf_{G \in \calP(\epsilon)}T(G) .
\eeq
Then $b_1 (\epsilon) = \max\{-b_-(\epsilon), b_+(\epsilon) \}$ (because $T(F) = 0$).
Define
\beq
    \lambda(t; G) = \EE_G \psi(X-t) = \int \psi(x-t) dG(x).
\eeq
Because $\psi$ is monotonic, $\lambda$ is decreasing in $t$, but not necessarily strictly
decreasing, so the solution of $\lambda(t;G) = 0$ is not necessarily unique.
Define
\beq
    T^*(G) = \sup \{ t : \lambda(t;G) > 0 \}
\eeq
and
\beq
    T^{**}(G) = \inf \{ t : \lambda(T;G) < 0 \}.
\eeq
Then $T^*(G) \le T(G) \le T^{**}(G)$.
Note that $\lambda(t; G)$ increases if $G$ is made stochastically larger.
The stochastically largest element of $\{ G: d_L(F, G) \le \epsilon \}$ is
\beq
    F_1(x) = (F(x - \epsilon) - \epsilon)_+ =
    \left \{
    \begin{array}{ll}
        0, & x \le x_0 + \epsilon \\
        F(x-\epsilon) - \epsilon & x > x_0 + \epsilon ,
    \end{array}
    \right .
\eeq
where $x_0$ solves $F(x_0) = \epsilon$.
(Assume that $x_0$ exists; the discontinuous
case introduces some additional bookkeeping.)
Note that this distribution puts mass $\epsilon$ at $x = \infty$.
For $G \in \calP(\epsilon)$,
\beq
    \lambda(t; G) \le \lambda(t; F_1) = \int_{x_0}^\infty \psi(x - t + \epsilon)dF(x)
    + \epsilon \psi(\infty).
\eeq
It follows that
\begin{eqnarray}
    b_+(\epsilon) &=& \sup_{G \in \calP(\epsilon)} T(G) \nonumber \\
    &= & T^{**}(F_1) \nonumber \\
    &= & \inf \{ t : \lambda(t; F_1) < 0 \}.
\end{eqnarray}
Note that
\beq
    \ell_+ = \lim_{t \rightarrow \infty} \lambda(t; F_1) =
    (1 - \epsilon) \psi(-\infty) + \epsilon \psi(\infty) .
\eeq
Provided $\ell_+ < 0$ and $\psi(\infty) < \infty$,
$b_+(\epsilon) < b_+(1) = \infty$.
Thus to avoid breakdown from above we need
\beq \label{eq:MUpBreakdown}
    \frac{\epsilon}{1-\epsilon} <  -\frac{\psi(- \infty)}{\psi(\infty)}.
\eeq
We can calculate $b_-(\epsilon)$ in the same way:
the stochastically smallest element of $\calP_(\epsilon)$ is
\beq
    F_{-1} = (F(x + \epsilon) + \epsilon) \wedge 1 =
    \left \{
        \begin{array}{ll}
             F(x + \epsilon) + \epsilon, & x \le x_1 - \epsilon \\
            1, & x > x_1 - \epsilon ,
        \end{array}
    \right .
\eeq
where $x_1$ solves $F(x_1) = 1- \epsilon$.
This distribution assigns mass $\epsilon$ to $x = -\infty$.
We have
\beq
    \lambda(t; G) \ge \lambda(t; F_{-1}) = \epsilon \psi(-\infty) +
    \int_{-\infty}^{x_1}\psi(x - t - \epsilon)dF(x).
\eeq
Thus
\begin{eqnarray}
    b_-(\epsilon) &=& \inf_{G \in \calP(\epsilon)} T(G) \nonumber \\
    &= & T^{*}(F_{-1}) \nonumber \\
    &= & \sup \{ t : \lambda(t; F_{-1}) > 0 \}.
\end{eqnarray}
Note that
\beq
    \ell_- = \lim_{t \rightarrow -\infty} \lambda(t; F_{-1}) =
    \epsilon \psi(-\infty) + (1-\epsilon) \psi(\infty) .
\eeq
To avoid breakdown from below,
we need  $\psi(- \infty) > - \infty$ and $\ell_- > 0$, which leads to
\beq
    \frac{\epsilon}{1-\epsilon} >  -\frac{\psi(\infty)}{\psi(-\infty)}.
\eeq
Combining this with \ref{eq:MUpBreakdown} gives
\beq
    -\frac{\psi(\infty)}{\psi(-\infty)} < \frac{\epsilon}{1-\epsilon}
    <  -\frac{\psi(- \infty)}{\psi(\infty)}.
\eeq
Define
\beq
    \eta \equiv \min \left \{ - \frac{\psi(-\infty)}{\psi(\infty)}, -
    \frac{\psi(\infty)}{\psi(-\infty)} \right \}.
\eeq
The breakdown point is then
\beq
    \epsilon^* = \frac{\eta}{1 + \eta} .
\eeq
The maximum possible value
$\epsilon^*  = 1/2$ is attained if $\psi(\infty) = - \psi(-\infty)$.
The breakdown point is $\epsilon^* = 0$ if $\psi$ is unbounded.

This calculation also shows that if $\psi$ is bounded and $\lambda(t; F)$ has a unique
zero at $t = T(F)$, then $T$ is continuous at $F$; otherwise, $T$ is not continuous at
$F$.

Things are much more complicated for non-monotone functions $\psi$, 
including ``redescending influence functions,'' which we shall examine presently.

\subsubsection{Minimax Properties for location estimates}
It is straightforward to find the minimax bias location estimate for symmetric
unimodal distributions; the solution is the sample median (see Huber, \S4.2).
Minimizing the maximum variance is somewhat more difficult.
Define
\beq
    v_1 (\epsilon) = \sup_{G \in \calP(\epsilon)} A(G,T),
\eeq
where
$A(G,T)$ is the asymptotic variance of $T$ at $G$.
Assume the observations are iid $G( \cdot - \theta)$.  The
shape varies over the family $\calP(\epsilon)$; the parameter varies over the
reals.
Such families are not typically compact in the weak topology.  Huber uses
the {\em vague} topology to surmount this problem.
The {\em vague} topology is the weakest topology on the set $\calM_+$ of
sub-probability measures for which $F \rightarrow \int \psi dF$
is continuous for all continuous functions $\psi$ with compact support.
(A subprobability measure can have total mass less than one, but is otherwise the same
as a probability measure.)
Because $\bfR$ is locally compact, $\calM_+$ is compact.

Define $F_0$ to be the distribution in $\calP(\epsilon)$ with smallest
Fisher information
\beq
    I(G) = \sup_{\psi \in \calC_K^1} \frac{(\int \psi' dG)^2}{\int \psi^2 dG},
\eeq
where $\calC_K^1$ is the set of all compactly supported,
continuously differentiable functions $\psi$ s.t. $\int \psi^2 dF > 0$.
This extends the definition of the Fisher information beyond measures that
have densities; in fact,
$I(F) < \infty$ iff $F$ has an absolutely continuous density w.r.t. Lebesgue measure,
and $\int (f'/f)^2 f dx < \infty$.

\noindent
{\bf Proof.} (Following Huber, pp78ff.)
By assumption, $\int \psi^2 dx < \infty$.
If   $\int (f'/f)^2 f dx < \infty$,
\begin{eqnarray}
    \left ( \int \psi' f dx \right )^2 & = &
    \left (\psi f |_{-\infty}^\infty - \int \psi \frac{f'}{f} f dx \right )^2
    \nonumber \\
    &=& \left ( \int \psi \frac{f'}{f} f dx \right )^2
    \nonumber \\
    &\le & \left ( \int \psi^2 f dx\right ) \left ( \int  \left ( \frac{f'}{f}\right )^2
        f dx \right ) ,
\end{eqnarray}
by the weighted Cauchy-Schwarz inequality.
Thus $I(F) \le \int (f'/f)^2 f dx < \infty$.
Now suppose $I(F) < \infty$.
Then  $L(\psi)  = -\int \psi' dF$ is a bounded
linear functional on the (dense) subset $\calC_K^1$ of $L_2(F)$, the Hilbert
space of square-integrable functions w.r.t. $F$.
By continuity, $L$ can be extended to a continuous linear functional on all
of $L_2(F)$.  By the Riesz Representation Theorem, there then exists a function
$g \in L_2(F)$ such that for all $\psi \in L_2(F)$,
\beq
    L\psi = \int \psi g dF.
\eeq
Clearly, $L1 = \int g dF = 0$.
Define
\beq
f(x) \equiv \int_{y < x} g(y) F(dy) = \int 1_{y < x} g(y) F(dy).
\eeq
By the Cauchy-Schwarz inequality,
\beq
    | f(x) |^2 \le (\int 1_{y <x}^2 F(dy) ) (\int g(y)^2 F(dy) )  =
    F((-\infty, x)) \int g^2 F(dy),
\eeq
which tends to zero as $x \rightarrow - \infty$;
$|f(x)|$ also tends to zero as $x \rightarrow \infty$.
For $\psi \in \calC_K^1$,
\beq
    - \int \psi'(x) f(x) dx = - \int_{y < x} \int \psi'(x) g(y) F(dy) dx =
    \int \psi(y) g(y) F(dy) = L\psi
\eeq
by Fubini's theorem.
Thus the measure $f(x) dx$   and the measure $F(dx)$ give the same linear functional
on derivatives of functions in $\calC_K^1$. This set is dense in $L_2(F)$, so
they define the same measure, and so $f$ is a density of $F$.
We can now integrate the definition of the functional $L$ by parts to show that
\beq
    L\psi = - \int \psi' f dx = \int \psi \frac{f'}{f} f dx.
\eeq
Thus
\beq
    I(F) = \| L \|^2 = \int g^2 dF = \int \left ( \frac{f'}{f} \right )^2 f dx.
\eeq

The functional $I(G)$ is lower-semicontinuous with respect to the vague topology,
so $I(G)$ attains its infimum on any vaguely compact set.
Furthermore, $I(G)$ is a convex function of $G$.

\begin{Theorem}
(Huber, Proposition 4.5)
Let $\calP$ be a set of measures on $\bfR$.
Suppose $\calP$ is convex, $F_0 \in \calP$ minimizes $I(G)$ over $\calP$,
$0 < I(F_0) < \infty$, and the set where the density $f_0$ of $F_0$ is strictly positive
is (a) convex and (b) contains the support of every distribution in $\calP$.

Then $F_0$ is the unique minimizer of $I(G)$ over $\calP$.
\end{Theorem}

The reciprocal of $I(F_0)$ lower-bounds the (worst) asymptotic variance of
any estimator over all $G \in \calP$, so if one can find an estimator whose
asymptotic variance is $1/I(F_0)$, it is minimax (for asymptotic variance).

Finding $F_0$ can be cast as a variational problem; see Huber, \S 4.5.

The least-informative distributions in neighborhoods of the normal
tend to have thinner tails than the normal.  If one believes that outliers might be
a problem, it makes sense to abandon minimaxity in favor of estimators that do
somewhat better when the truth has thicker tails than the normal.
That leads to considering {\em redescending influence functions}, for which
$\psi = 0$  for $x$ sufficiently large.

One can develop ``minimax'' estimators in this restricted class.
For example, we could seek to minimize the asymptotic variance subject to
$\psi(x) = 0$, $|x| > c$.
For the $\epsilon$-contamination neighborhood of a normal, the minimax $\psi$
in this class is
\beq
\psi(x) = - \psi(-x) = \left \{ \begin{array}{ll}
x, & 0 \le x \le a \\
b \tanh \left ( \frac{b(c-x)}{2} \right ) & a \le x \le c \\
0, & x \ge c.
\end{array}
\right .
\eeq
The values of $a$ and $b$ depend on $\epsilon$.

Other popular redescending influence functions include Hampel's piecewise linear influence
functions:
\beq
    \psi(x) = - \psi(-x) = \left \{
        \begin{array}{ll}
            x, & 0 \le x \le a \\
            a & a \le x \le b \\
            a\frac{c-x}{c-b} & b \le x \le c \\
            0, & x \ge c,
        \end{array}
    \right .
\eeq
and Tukey's ``biweight''
\beq
    \psi(x) = \left \{
    \begin{array}{ll}
        x(1-x^2)^2, & | x | \le 1 \\
        0, & |x | > 1.
    \end{array}
\right .
\eeq
A complication in using redescending influence functions is that scaling (some
form of Studentizing) is much more important for them to
be efficient than it is for monotone influence functions.
The slope of the influence function in the descending regions also can
inflate the asymptotic variance (recall that $(\int \psi'  dF)^2$ is in the
denominator of $A(F, T)$).


\subsection{Estimates of Scale}
We require that a scale estimate $S_n$ be equivariant under changes of scale, so that
\beq
    S_n (aX) = a S_n(X), \;\; \forall a > 0 .
\eeq
It is common also to require that a scale estimate be invariant under sign
changes and translations, so that
\beq
    S_n(-X) = S_n(X) = S_n(X+b{\bf 1}),
\eeq
where $b \in \bfR$ and ${\bf 1}$ is an $n$-vector of ones.
The most common need for a scale estimate is to remove scale as a nuisance
parameter in a location estimate, by Studentizing.

It turns out that the bias properties of a scale estimate are more important for
studentizing than the variance properties. That leads to considering something
involving the median deviation.
The most widely used robust scale estimator is
the median absolute deviation (MAD).
Let $M_n(x)$ be the median of the list of the elements
of $x$: $\med\{x_j\}_{j=1}^n$.
Then
\beq
    \MAD_n (x) = \med \{ | x_j - M_n(x) | \}_{j=1}^n.
\eeq
The breakdown point of the MAD is $\epsilon^* = 1/2$.
Typically, the MAD is multiplied by 1.4826 ($1/\Phi^{-1}(3/4)$)
to make its expected value unity for a standard normal.


\subsection{Robust Regression}
This section follows Hampel {\em et al.\/} (1986), Chapter 6, rather closely.
Suppose we have a linear statistical model: we observe $\{(X_j, Y_j)\}_{j=1}^n$
iid with $Y_j \in \bfR$, $X_j \in \bfR^p$ ($p \ge 1$), and
\beq
    Y = X\theta + \epsilon,
\eeq
where $Y = (Y_j)_{j=1}^n$, $X$ is an $n$ by $p$ matrix with entries
$X_{jk} = (X_j)_k$, $1 \le j \le n$, $1 \le k \le p$,
$\theta \in \Theta \subset \bfR^p$ is unknown, $\Theta$ is an open convex subset of
$\bfR^p$, and $\epsilon \in \bfR^n$ is a
vector of iid noise with mean zero.
We assume that $\epsilon_j$ is independent of $X_j$, that the common distribution
of the noise components is symmetric about zero and continuous.
Let $K$ be the common distribution of $\{X_j\}$, and let $k$ be the density of $K$
with respect to Lebesgue measure on $\bfR$.
The joint density of $(X_j, Y_j)$ with respect to Lebesgue measure on $\bfR^2$ is
\beq
    f_\theta(x, y) = \sigma^{-1} g\left ( (y - x\theta)/\sigma\right ) k(x).
\eeq
Let $F_\theta(x, y)$ be the corresponding measure.
For now, we take $X$ and $Y$ to be random (we don't condition on $X$).

The least-squares (LS) estimate of $\theta$ is
\beq
    \hat{\theta}_{LS} \equiv
    \arg \min_{\gamma \in \Theta} \sum_{j=1}^n \left ( (y_j - x_j\gamma) / \sigma \right )^2.
\eeq
The least squares estimate minimizes the 2-norm of the residuals.
The Gauss-Markov theorem says that if $\EE \epsilon_j = 0$, $1 \le j \le n$, and
$cov(\epsilon) = \sigma^2 I$, then $b\hat{\theta}_{LS}$ is
the unique minimum variance unbiased linear estimator
of any estimable linear functional $b\theta$ of $\theta$.
The restriction to linear estimators is severe---for example,
the MLE is nonlinear for many distributions.
When the error distribution is normal, the minimum variance estimate
is in fact linear, but otherwise, there can be a large loss of efficiency in
restricting to linear estimators.

Moreover, a single outlier can influence the least-squares estimate arbitrarily---the least-squares
estimate is neither robust nor resistant.
(See {\tt http://statistics.berkeley.edu/users/stark/Java/Correlation.htm\/}
for an interactive demonstration.)

Huber (1973, Robust regression: asymptotics, conjectures, and Monte Carlo.
{\em Ann. Stat.\/}, {\em 1\/}, 799--821.)
proposed using least squares with weights on the residuals computed iteratively
to reduce the influence of extreme observations.
The weights are
\beq
    w_j = \min (1, c/|r_j|),
\eeq
where $c$ is a positive constant and $r_j$ is the $j$th residual.
This is a special case of the estimator
\beq
    \hat{\theta}_{\rho} \equiv \arg \min_{\gamma \in \Theta}
        \sum_{j=1}^n \rho \left ( (y_j - x_j\gamma)/\sigma \right ),
\eeq
where $\rho$ is a nonnegative function.
The case $\rho(r) = r^2/2$ is least squares.
The case $\rho(r) = \rho_c(r) = w r^2$ yields the MLE when the errors have density proportional
to $\exp(-\rho_c(r))$.

The case $\rho(r) = |r|$ is minimum $L_1$ estimation, which is resistant.
Minimum $L_1$ regression can also be performed using linear programming.
We shall use linear programming in our discussion of density estimation
later, so I'll take a bit of space here to set up the $L_1$ regression problem as
a linear program.

Linear programming solves the problem
\beq
    \min_{z: z \ge 0, Az = b} \ell z,
\eeq
where $z, \ell \in \bfR^k$, $A$ is an $m$ by $k$ matrix, and $b \in \bfR^m$.
(There are other canonical forms for linear programs, such as
$\min_{z: Az \le b} \ell z$.)
Linear programming and its infinite-dimensional extensions play a central
role in game theory and minimax statistical estimation.

Linear programming involves minimizing a convex functional over a
convex set, so any local minimum attains the global minimal value.
However, neither the constraint set nor the objective functional
is strictly convex, so in general more than one parameter vector
attains the global minimal value of the objective functional.
The nonnegativity constraint and the linear equality constraints
restrict the solution to lie within a convex {\em polytope\/}, which
is analogous to a convex polygon in $\bfR^2$: it is the intersection
of a collection of closed half-spaces.
The fundamental theorem of linear programming says that if the
linear program is consistent (if there is some vector $z$ that
satisfies all the constraints), then some vertex of the polytope
attains the global minimal value of $\ell$.

We seek to cast the problem
\beq
    \min_{\gamma \in \bfR^p} \sum_{j=1}^n | X_j \gamma - y_j|
\eeq
as a linear program.
The minimizing $\gamma$ is the $L_1$ regression estimate of $\theta$.
Define
\beq
    A = [X\;\; -X\;\; I\;\; -I],
\eeq
\beq
    b = y,
\eeq
\beq
    z = [f\;\; g\;\; d\;\; e]^T,
\eeq
(with $f, g \in \bfR^p$ and $d, e \in \bfR^n$), and
\beq
    \ell = [\bf0\;\; \bf1]
\eeq
(with $\bf0 \in \bfR^{2p}$ and $\bf1 \in \bfR^{2n}$).
Then
\beq
    Az = b \Longleftrightarrow X(f-g) + d - e = y,
\eeq
and minimal value of $\ell z$ minimizes
\beq
    \sum_{j=1}^n (d_j + e_j).
\eeq
Note that if, for some $j$, both $d_j$ and $e_j$ are greater
than zero, then replacing the smaller of the two with 0
and the larger of the two with $|d_j - e_j|$ reduces
$\ell z$ but does not change the value of $Az$.
Thus for the optimal $z$, either $d_j=0$ or $e_j=0$ (or both) for each $j$.
Therefore, at the optimum, $|d_j - e_j| = d_j + e_j$; i.e.,
\beq
    | X_j(f-g) - y_j | = d_j + e_j,
\eeq
and
\beq
    \sum_{j=1}^n | X_j(f-g) - y_j | = \ell z.
\eeq
The $L_1$ regression estimate of $\theta$ is clearly $\hat{\theta}_{L_1} = f - g$,
where $f$ and $g$ are the first $p$ and second $p$ components of $z^*$,
the solution to the linear programming problem above.

Linear programming also can solve $L_\infty$ regression problems:
\beq
    \min_{\gamma} \max_j | X_j\gamma - y_j|.
\eeq
$L_\infty$ regression is extremely non-resistant.

The most popular algorithm for solving linear programs is
the simplex algorithm.
The simplex algorithm starts with a feasible point, then moves
from vertex to vertex of the polytope defined by the linear
constraints, continually decreasing the value of $\ell z$.
(The fundamental theorem of linear programming guarantees
that there is a vertex that attains the optimal value.)
At each vertex, the algorithm checks the Kuhn-Tucker conditions, which
say that at an optimal solution, every direction of descent
points out of the constraint set.

We continue with the discussion of robust regression for a more general weight
function $\rho$.
If $\rho$ is differentiable with derivative $\partial \rho(r)/\partial r = \psi(r)$,
then $\hat{\theta}_\rho$ can be characterized as
a stationary point
\beq
    \sum_{j=1}^n \psi \left ( ( y_j - x_j \hat{\theta}_\rho )/\sigma \right ) x_j = 0.
\eeq
(Note that for $L_1$ regression, $\rho$ is not differentiable at 0.)

Typically, $\sigma$ is not known.
It could be estimated together with $\theta$; I think it is more common to
use the MAD of the residuals, estimated iteratively for a few iterations, then
``frozen'' for the rest of the iterations in the estimation of $\theta$.

For the Huber estimate using the weight function $w$, the influence of the residual
is bounded, but the influence of the $x$-coordinate of a datum is not bounded.
(See Fig. 1, p.~314, in {\em Hampel et al.\/})
If we want to use the errors-in-variables model, that should be considered as well.
The influence function can be factored into the influence of the residual (IR)
and the influence of position in factor space (IP)--analogous to the leverage
of the point.

{\em Hampel et al.\/} (1986, ch.~6.3, p.~315) define an $M$-estimator $T_n$ for linear
models as the solution to the equation
\beq
    \sum_{j=1}^n \eta \left (X_j, (y_j - X_jT_n)/\sigma \right ) x_j = 0,
\eeq
where $\eta: \bfR^p \times \bfR \rightarrow \bfR$ is suitably well behaved:
\begin{enumerate}
    \item
        For every $x \in \bfR^p$, $\eta(x, \cdot)$ is continuous except possibly
        at a finite set, where it has finite left and right limits.
    \item
        For every $x \in \bfR^p$, $\eta(x, r)$ is an odd function of $r$, and is
        nonnegative for nonnegative $r$.
    \item
        For every $x \in \bfR^p$, the set of points at which
        $\eta'(x, r) \equiv \partial \eta(x, r)/\partial r$ fails to exist or fails to
        be continuous, is a finite set.
    \item
        $M \equiv \EE \eta'(x, r) x x^T$ exists and is a nonsingular matrix.
    \item
        $Q \equiv \EE \eta^2(x, r) x x^T$ exists and is a nonsingular matrix.
\end{enumerate}
This is more restrictive than the general $M$-estimation set up: the function
$\psi(X_j, y_j, \theta)$ that this implicitly defines has the same direction
as $X_j$, and depends on $y_j$ only through the residual $y_j - X_j \theta$.

According to {\em Hampel et al.\/}, the functions $\eta$ that have
been proposed in the literature all factor into a weight that depends on
$x$ and function of the residual weighted by a different function of $x$:
\beq
    \eta(x, r) = w(x) \psi(r v(x)),
\eeq
where $w: \bfR^p \rightarrow \bfR^+$, $v: \bfR^p \rightarrow \bfR^+$,
and $\psi: \bfR \rightarrow \bfR$.
Three different types of $\eta$ functions are sketched on p.~322 of
{\em Hampel et al.\/}:
\begin{enumerate}
    \item The Huber estimator discussed above takes $w(x)=v(x)=1$:
        \beq
            \sum_{j=1}^n \psi_c(r_j) X_j = 0.
        \eeq
        This estimator does not limit the influence of position.
    \item The Mallows estimator takes $v(x)=1$:
        \beq
            \sum_{j=1}^n \psi_c(r_j) w_j X_j = 0.
        \eeq
        This estimator limits the leverage of a datum regardless of
        the size of the residual at that design point.
    \item The Hampel-Krasker estimator takes $w(x) = 1/v(x)$:
        \beq
            \sum_{j=1}^n \psi_c(r_j/u_j) u_j X_j = \sum_{j=1}^n \psi_{cu_j}(r_j) X_j = 0.
        \eeq
\end{enumerate}
Figure 2 on p.~323 of {\em Hampel et al.\/} illustrates the difference in the
fitted regression function for these three estimators for the bivariate case with one
outlier with large leverage.

Below is a collection of Matlab routines that solve iteratively reweighted
least squares using robust estimates of scale.
The routines do not bound the influence of position; they just bound
the influence of the residuals.
(The extension to weight by position is straightforward.)
Included are Matlab routines to calculate Tukey's Biweight and the weight
function for minimum $L_1$ regression by reweighting.
At the end is a routine for minimum $L_1$ regression using linear
programming.
I wrote this code in 1997 and would write it differently now $\ldots$

\begin{verbatim}

function [ b, res, wssr, it ] = irwls(y,X,wfunc,param,start)
% irwls    iteratively re-weighted least squares
%          Can solve "robust" regression problems using
%          robust weights.
% Usage:
%        [b, res, wssr, it ] = irwls(y,X,wfunc,param,start)
%
% Arguments:
%       y  n-vector of data
%       X      nxp design matrix
%       wfunc  the name of the weight function subroutine, passed as a string
%               W = wfunc(res, scale, param) must take three arguments:
%           res     vector of residuals
%           scale   resistant estimate of scale (MAD/.6745)
%           param   array of particular parameters passed by caller
%           wfunc must return a vector W of the same dimension as res
%       param  a vector of parameters to pass to wfunc.
%           DEFAULT = 1.
%       start  a flag for what to use as a starting model to get
%       the estimate of scale that is passed to the weight
%       function. To get the estimate of scale, the algorithm
%       uses the median absolute deviation (MAD) of the residuals,
%       standardized to that of the normal distribution.
%       If start = 'ols', the median absolute deviation of
%       the residuals from the ordinary least-squares regression
%       function are used.
%       If start = 'median', the median absolute deviation
%       of the residuals from the median are used, completely
%       ignoring the possible variation of the "true" function.
%       If start = 'one-step', the first estimate of scale
%       is the same as for 'median', but the scale estimate is
%       updated based on the residuals from the first step
%       of the least squares fit using that scale (the scale estimate
%       is only updated once, after the first wls).
%       DEFAULT = 'ols'
%
% Output:
%       b      p-vector of fitted coefficients
%       res    n-vector of final residuals
%       wssr   final sum of squared weighted residuals
%       it     number of iterations taken

% P.B. Stark    stark@stat.berkeley.edu
% 9 July 1997

chgTol = 0.001;                 % tolerance for convergence
                                % of coefficients
maxit = 500;                    % maximum number of iterations

% check dimensions
    [n,p] = size(X);
    [ny, py] = size(y);

    if (n ~= ny),
        error('number of rows of y ~= number of rows of X')
    end

    if (py ~= 1),
        error('y has more than one column')
    end

    if (nargin < 3),
        param = 1;
        start = 'ols';
    elseif (nargin < 4),
        start = 'ols';
    end

    if (~(strcmp(start,'ols') | strcmp(start,'median') | ...
        strcmp(start,'one-step'))),
        error(['start = ', start, ' not supported'])
    end

    %
    if (strcmp(start,'ols')),
        W = ones(n,1);              % initial weight matrix
        [b, res, wssr] = wls(y,X,W);        % initial regression
        scale = 1.483*median(abs(res-median(res)));
    elseif (strcmp(start,'median')),
        res = y - median(y);
        scale = 1.4843*median(abs(res-median(res)));
        W = diag(sqrt(feval(wfunc,res,scale,param)));
        [b, res, wssr] = wls(y, X, W);
    elseif (strcmp(start,'one-step')),
        res = y - median(y);
        scale = 1.4843*median(abs(res-median(res)));
        W = diag(sqrt(feval(wfunc,res,scale,param)));
        [b, res, wssr] = wls(y, X, W);
        scale = 1.4843*median(abs(res-median(res)));
    end

% loop
    relchg = 1;                         % relative change in coefficients
    it = 0;
    while (relchg > chgTol & it < maxit),       % stop when the relative change in
                                        % regression coefficients between
                                        % iterations is small enough
        lastB = b;
        it = it + 1;
            W = diag(sqrt(feval(wfunc,res,scale,param)));
        [b, res, wssr] = wls(y, X, W);
        relchg = max( abs( b - lastB ) ./ abs(b));
    end
    if (it == maxit ),
        error(['maximum number of iterations (' num2str(maxit) ...
             ') exceeded'])
    end
return


function [b, res, wssr] = wls(y,X,W)
%  wls  Multiple linear regression using weighted least squares.
%
%  Usage:
%   [b, res, wssr] = wls(y,X,W) returns the p-vector b of
%       regression coefficients such that
%           (Xb - y)' W'W (Xb - y)    (**)
%       is minimal, along with the residual n-vector
%           res = y - Xb,
%       and the weighted sum of squared residuals
%           wssr = (Xb - y)'  W  (Xb - y).
%  Arguments:
%   y   n-vector of observations
%   X   nxp design matrix
%   W   nxn positive semi-definite square matrix of weights,
%       or a vector of nonnegative weights.
%       Note that W is the square-root of what one would think
%       of as the data error covariance matrix.
%
%
%  Algorithm:   Uses the QR decomposition as follows.
%       Define Z = WX and d = Wy. Then (**) can
%       be written as an OLS (ordinary least squares)
%       problem: minimize
%           || Zb - d ||^2.
%       If [Q, R] is the QR decomposition of Z,
%       stationarity yields
%           2Z'(Zb - d) = 0,
%       or  Z'Zb = Z'd.
%       The QR decomposition gives
%           R'Rb = R'Q'd; for nonsingular R,
%           Rb = Q'd.
%       Matlab can solve this triangular system stably using
%           b = R\(Q'd).
%

    if  nargin < 3,
        error('wls requires three input arguments.');
    end

% Check that X, y, and W have compatible dimensions
    [n,p] = size(X);
    [ny,py] = size(y);
    [nW,pW] = size(W);
    dimW = max(nW, pW);
    minDimW = min(nW, pW);
    %
    if (n ~= ny),
        error( 'The number of rows in y must equal the number of rows in X' );
    elsif ( py ~= 1 ),
        error( 'y must be a vector, not a matrix' );
    elsif ( minDimW > 1 & nW ~= pW ),
        error( 'W must be a square matrix or a vector' );
    elsif ( dimW ~= n ),
        error( 'W must have the same number of rows as X and y' );
    elsif ( minDimW == 1 & min(W) < 0 ),
        error( 'W must be nonnegative' );
    end

% build the weighted design matrix and data

    if ( minDimW == 1),
        W = diag(W);        % in case W is a vector
    end

    Z = W*X;
    d = W*y;

    [Q, R] = qr(Z, 0);
    b = R\(Q'*d);

    yHat = X*b;             % predicted data values
    res = y - yHat;         % residuals
    wssr = norm(W*res);     % weighted sum of squared residuals

return;


function W = l1wght(res, scale, param)
% function W = l1wght(res, scale, param)
%
% computes the weight function for minimum l_1 regression
% for iteratively reweighted least squares: scale/|res|
%
% arguments:
%       res:    vector of residuals
%       scale:  robust estimate of scale such as MAD
%       param:  not used--present only for consistent
%           calling signature with other weights
%
% returns:
%       W:  the vector of l_1 weights
%
%
% P.B. Stark  stark@stat.berkeley.edu
% 11 July 1997.
    thresh = eps^(1/3);

    unit = ones(size(res))*thresh;
    res(abs(res) <= thresh ) = unit(abs(res) <= thresh);
    W = (abs(res)).^(-1);
return;

function W = biwght(res, scale, param)
% function W = biwght(res, scale, param)
%
% computes Tukey's biweight function for robust regression
%
% biwght(x) =   (1-x^2/param^2)^2 , |x| < param; 0, |x| >= param.
%
% arguments:
%       res:    vector of residuals
%       scale:  robust estimate of scale, such as MAD
%       param:  parameter of the biweight function.
%           8 is a reasonable number, and is the
%           default.
%
% returns:
%       W:  the vector of biweight weights
%
%
% P.B. Stark  stark@stat.berkeley.edu
% 11 July 1997.

    p = 8;
    if (nargin == 3),
        p = param;
    end

    res = res/scale;
    W = (ones(size(res)) - res.^2/p^2).^2 .* (abs(res) < p);

return;


\end{verbatim}

Here is an implementation of $L_1$ regression using Matlab's linear programming
routine.
Note that this requires the Optimization Toolbox.

\begin{verbatim}
function [ b, res, msft ] = l1rgres(y, X, verb, alg)
% l1rgres   minimum l_1 misfit regression using linear programming
%
%   WARNING: Matlab's lp implementation is unstable!
%
% Usage:
%        [ b, res, msft ] = l1rgres(y, X, verb, alg)
%
% Input:
%        y  n-vector of data
%        X      nxp design matrix
%        verb   if verb = -1, LP runs silently.
%        alg    string for the linear programming algorithm to use.
%            If alg == 'lp', uses the MATLAB Optimization toolbox
%            LP.M routine.
%            Defaults to 'lp'.
%
%
% Output:
%        b      p-vector of fitted coefficients
%        res    n-vector of final residuals
%        msft   l_1 misfit of solution
%
% P.B. Stark    stark@stat.berkeley.edu
% 10 July 1997


%
% check dimensions
    [n,p] = size(X);
    [ny, py] = size(y);

    if (n ~= ny),
        error('number of rows of y ~= number of rows of X')
    end

    if (py ~= 1),
        error('y has more than one column')
    end

% output level?
    talk = 0;
    if (nargin >= 3),
        talk = verb;
    end;

%
% set up slack variables and their bounds
%
% want  to minimize
%   sum_{j=1}^n ( pos_j + neg_j )
% subject to
%   y = Ab + pos - neg
%   0 <= pos <= infty
%   0 <= neg <= infty
%
% what to do depends on the linear programming algorithm used.
%
    lpalg = 'lp';
    if (nargin == 4),
        lpalg = alg;
        if (~strcmp(lpalg,'lp') ),
            error(['1: the algorithm ' lpalg ' is not supported'])
        end;
    end;

%
% use ols to get a cheap feasible starting model
    [xols, olsres, rssols] = wls(y, X, ones(n,1));

% the residuals give the values of the slacks
    pos = zeros(size(olsres));
    neg = pos;
    pos(olsres > 0) = olsres(olsres > 0);
    neg(olsres < 0) = -olsres(olsres < 0);
    clear olsres;               % free the memory

% now the lp phase
    if (talk >= 0),
        disp('starting LP phase')
    end

% cases, by LP algorithm invoked
%
    if (strcmp(lpalg,'lp')),
        lpn = p + 2*n;          % number of variables, including slacks
        lpm = n;            % total number of constraints
        neq = n;            % number of equality constraints
        f = [zeros(1,p) ones(1,2*n)];   % objective vector (a row vector)
        vlb = [-inf * ones(p,1); ...
                zeros(2*n,1) ];     % lower bounds on variables
        vub = inf * ones(lpn,1);    % upper bounds
        A = [X ones(n) -ones(n)];   % add the slacks to the constraints
                        % this gets HUGE quickly!
        x0 = [xols; pos; neg];
        lpx = lp(f, A, y, vlb, vub, x0, neq, talk);
        msft = f * lpx;
        b = lpx(1:p);           % extract solution
        res = y - X*b;
    else
        error(['impossible error: algorithm ', alg, ' not supported'])
    end;

return

\end{verbatim}


\end{document}


